\documentclass[t]{beamer}
\setlength{\parskip}{5pt}
%\usetheme{Madrid}  

\usepackage[UKenglish]{babel}
\usepackage[UKenglish]{isodate}
\cleanlookdateon

\newtheorem{remark}{Remark}

\def\le{\leqslant}
\def\ge{\geqslant}

\def\Z{\mathbb{Z}}
\def\N{\mathbb{N}}
\def\R{\mathbb{R}}
\def\C{\mathbb{C}}
\def\Q{\mathbb{Q}}

\title{CS3230 Tutorial 1}
\author{Deng Tianle (T15)}
\date{22 August 2025}

\begin{document}

\frame{\titlepage} 

\begin{frame}{Introduction}
  This is Tutorial Group 15 for CS3230.

  \par My name is Deng Tianle: 
  \begin{itemize}
    \item Year 3 Computer Science and Mathematics (DDP)
    \item First-time TA :)
    \item My JC: Raffles Institution (Y5-6)
    \item I was from Shenzhen, China
  \end{itemize}
\end{frame}
\begin{frame}{Admin}
  Everyone needs to present $3$ times to obtain the $3\%$ tutorial participation marks. (I believe that beyond the presentation, there is no obligation to attend tutorial. But of course, you are encouraged to attend).
  \begin{align*}
    &\text{We have $21$ people} \\ 
    \implies& \text{$63$ presentations} \\
    \implies& \text{we should have around $6$ presentations per class}.
  \end{align*}
  \begin{remark}
    In later slides I use $P$ to denote presentations, e.g. $P3$ means that the third presenter of the day will present on (possibly a part) of this question. 
  \end{remark}
\end{frame}

\begin{frame}{Agenda}
  This tutorial is about asymptotic notations: $O, \Omega, \Theta, o, \omega$.  
  \begin{itemize}
    \item Analogy with $\le, \ge, =, <, >$: Q2
    \item Computation using limit: Q1 (I rearranged because we can use Q2 here)
    \item Practical computation, relation between common functions like $\log$, polynomial, $\exp$, factorial: Q3-5
    \item LeetCode question on removing duplicates (if time permits; I will run the algorithm on the board to show the idea)
  \end{itemize}
\end{frame}
\begin{frame}
  Let $\N$ denote the set of positive integers. We consider functions $f: \N \to \N$. Let the class of all such functions be $C$. 
  \begin{definition}
    Let $f$ and $g$ be functions $\N \to \N$. 
    We say that $f(n) \in \Theta(g(n))$ if there exist positive constants $c_1, c_2$ such that \[c_1g(n) \le f(n) \le c_2g(n)\] for sufficiently large $n$. We say that $f(n) \in \omega(g(n))$ if for all positive constants $c$, we have
    \[cg(n)<f(n)\] for sufficiently large $n$. 
  \end{definition}
  \begin{remark}
    All the precise definitions are on the tutorial sheet. I only want to caution that $\forall, \exists$ are considered informal shorthands in mathematics (excluding logic and set theory), it is preferred to spell them out in formal writings. 
  \end{remark}
\end{frame}
\begin{frame}{Q2: P1, 2, 3}
  Q2 says that $O, \Omega, \Theta, o, \omega$ behave very much like $\le, \ge, =, <, >$ respectively. We want to make this more precise. Recall that $C$ denotes the set of functions $\N \to \N$. From the reflexivity, transitivity and symmetry parts of Q2, we get:
  \begin{theorem}
    For $f, g \in C$, we define $f\sim g$ iff $f(n) \in \Theta(g(n))$. Then $\sim$ is an equivalence relation. \footnote{Note that in analysis, $\sim$ already has a meaning that is stronger than this; we do not consider that definition. }
  \end{theorem}
  Now we recall the following result from lecture:
  \[\Theta(g) = O(g)\cap \Omega(g).\]
  This means that 
  \[ f(n) \in \Theta(g(n)) \iff \left(\,f(n) \in O(g(n)) \quad \text{and}\quad f(n) \in \Omega(g(n))\,\right).\]
\end{frame}

\begin{frame}
  %Together with the facts in Q2, we get
  \begin{theorem}
    For $[f], [g] \in C/{\sim}$, we define $[f] \le [g]$ iff $f(n) \in O(g(n))$ iff $g(n) \in \Omega(f(n))$. Then $\le$ is a well-defined partial order on $C/{\sim}$. % 
  \end{theorem}
  \begin{proof}
    Well-definedness follows from lecture result and transitivity. Antisymmetry (if $[f]\le[g], [g]\le[f]$ then $[f]=[g]$) follows from lecture result. The rest follow from Q2.%The only nontrivial part left is that  
  \end{proof}
  \begin{theorem}
    If $f(n) \in o(g(n))$ i.e. $g(n) \in \omega(f(n))$, then $[f]<[g]$ (meaning, $[f] \le [g]$ but $[f] \ne [g]$).
  \end{theorem}
  \begin{proof}
    If $f(n) \in o(g(n))$ and $f(n) \in \Theta(g(n))$, then there is some $c>0$ such that 
    \[cg(n) \le f(n) < cg(n)\]
    for sufficiently large $n$, contradiction.
  \end{proof}
\end{frame}
\begin{frame}
  \begin{remark}
    \alert{It is not true that $[f]<[g] \implies f(n) \in o(g(n))$ (and similarly for $\omega$)}
  \end{remark}
  However, it turns out that $o$ itself induces naturally a strict partial order $<_o$ on $C/{\sim}$ which is `more selective' than $<$ in the sense that $[f]<_o[g] \implies [f]<[g]$ but not vice versa. 
  \begin{theorem}
    For $[f], [g] \in C/{\sim}$, we define $[f] <_o [g]$ iff $f(n) \in o(g(n))$ iff $g(n) \in \omega(f(n))$. Then $<_o$ is a well-defined strict partial order on $C/{\sim}$. 
  \end{theorem}
  \begin{remark}
    \alert{They are not total orders on $C/{\sim}$}. 
  \end{remark}
  I move the proof to the appendix. Exercise: give counterexamples for the two remarks above. (Hint: consider $g$ that `oscillates')
  %\par To reiterate about the second remark: you cannot show $o$ (or $\omega$) by using $O$ (or $\Omega$) and $\ne$, you will have to use limit or definition. 
\end{frame}

\begin{frame}{Q1: P4}
  The conclusions here are very important tools for computations (as we will see in Q3-5). We recall the definition of limit:
  \begin{definition}
    We say that $\lim_{n \to \infty}\phi(n) = l$ if for all $\epsilon>0$, there exists $N$ such that
    \[n \ge N \implies |\phi(n)-l|<\epsilon.\]
    We say that $\lim_{n \to \infty}\phi(n) = \infty$ if for all $M$, there exists $N$ such that
    \[n \ge N \implies \phi(n)>M.\]
  \end{definition}
  We would have someone presenting on the one for $O$. Then observe that by Q2, $\Omega$ follows from $O$, $\Theta$ follows from $O$ and $\Omega$, $\omega$ follows from $o$ (make sure you understand the proof in lecture!)
\end{frame}
\begin{frame}{Q3-5: P5, 6}
  We know from calculus/analysis that ($c, d$ are positive constants)
  \[[\log{n}] <_o [n^d] <_o [(1+c)^n] <_o [n!].\]
  Some useful facts that follow immediately from definition:
  \begin{itemize}
    \item $[cf(n)] = [f(n)]$ for constant $c>0$
    \item If $f(n)$ is a (finite) sum, then if a term of the sum has some other term $\ge$ it, it can be ignored. 
  \end{itemize}
  Most if not all of Q3-5 can be done using limits, order properties and such basic facts if you do not want to directly use definition. 
  \par I am asked to discuss this `hidden' variant of Q4: consider $2^{\log_4{n}}$, is it in $O(n)$? $\Omega(n)$? $\Theta(\sqrt{n})$ ? $\omega(n)$?
\end{frame}
\begin{frame}{Appendix}
  \begin{theorem}
    For $[f], [g] \in C/{\sim}$, we define $[f] <_o [g]$ iff $f(n) \in o(g(n))$ iff $g(n) \in \omega(f(n))$. Then $<_o$ is a well-defined strict partial order on $C/{\sim}$. 
  \end{theorem}
  \begin{proof}
    Well-definedness: if $[f_1]=[f_2]$ and $[f_2] <_o [g]$, then there exists $c_0>0$ such that for all $c>0$, we have
    \[f_1(n) \le c_0f_2(n) < c_0\frac{c}{c_0}g(n)=cg(n)\]
    for sufficiently large $n$, showing that $[f_1] <_o [g]$. Analogously, if $[g_1]=[g_2]$ and $[f] <_o [g_2]$, then $[f] <_o [g_1]$. 
    \par Irreflexivity: we have shown that if $[f] <_o [f]$ then $[f] \ne [f]$, contradiction. 
    \par Assymmetry: trivial exercise in CLRS
    \par Transitivity: from Q2.
  \end{proof}
\end{frame}
\begin{frame}{Q3}
  \begin{itemize}
    \item True: $3^{n+1}=3\cdot3^n \in \Theta(3^n)$ because we are just multiplying by constant. In particular, it is in $O(3^n)$. (If you want to show by definition, take $c=3$ and it goes through for all $n$). 
    \item False. I will explain this in two ways (there were some questions relating to this part, please feel free to clarify with me further): \begin{itemize}
      \item Use the theorems on analogy with orders that we have established (this allows you to see immediately whether it is true, and is good for MCQs): by limit, $[2^n]<_o[4^n]$, so $[2^n] < [4^n]$ and so it is not true that $[2^n] \ge [4^n]$ i.e. $4^n \in O(2^n)$. 
      \item Alternatively, you can use definition (this is good if you know it is false already and is asked to prove it in exam): to prove negation of $4^n \in O(2^n)$, you want to show that for all $c>0$ and for all $n_0$, there is some $n \ge n_0$ such that 
      \[2^n \cdot 2^n = 4^n > c \cdot 2^n.\]
      This is indeed true because for large enough $n$ we always have $2^n >c$. 
    \end{itemize}
  \end{itemize}
\end{frame}
\begin{frame}{Q3}
  \begin{itemize}
    \item True: We have 
    \[\frac{1}{2} n \le 2^{\log{n}-1} \le 2^{\lfloor \log n \rfloor} \le 2^{\log n} = n\]
    Hence $2^{\lfloor \log n \rfloor} \in O(n)$ (take $c=1$ if you want to show using definition) and $2^{\lfloor \log n \rfloor} \in \Omega(n)$ (take $c=1/2$)
    \item True: From binomial expansion of $(n+a)^i$, the dominant term is $n^i$ (all lower $n$ powers can be ignored), so this is in $\Theta(n^i)$. 
  \end{itemize}
\end{frame}
\begin{frame}{Q4}
   I will only discuss the hidden question. We have $2^{\log_4 n} = \sqrt{n}$ (there are many ways to see this depending on your high school background, e.g. use $\log_{2^2}\sqrt{n}^2 = \log {\sqrt{n}}$, or change of base $log_4{n} = \frac{\log{n}}{\log{4}} = \frac{\log n}{2}$) 
    \par Then for the same options shown, it is in $\Theta(\sqrt{n})$ and $O(n)$ but not in $\Omega(n)$ and hence not in $\omega(n)$ (all can be seen from $\sqrt{n} <_o n$)
\end{frame}
\begin{frame}{Q5}
  As explained in class, $\log(n^2)=2\log(n)$ so $[\log(n^2)]= [\log(n)]$. The rest follow from the facts shown in the slides before Appendix. 
  \par Final answer:
  \[[f_1]=[f_5]<_o [f_4] <_o [f_3] <_o [f_2]. \]
  For $[n!] >_o a^n$, one way to see this is that 
  \[n! \ge n(n-1)\dots(n/2) \ge (n/2)^{n/2}\]
  (no need to be too careful but odd/even case of $n$ since we can multiply $n!$ by constant anyway) and $\sqrt{n/2}$ exceeds $a$ for large $n$. 
\end{frame}

\begin{frame}{Q5}
  \[T(n)=4T(n/2)+\sqrt{n}\]
  Step 1: Guess the answer (without using Master theorem, because it may not always apply). Need some experience/intuition/luck, not 100\% methodological
  \par Possibility 1: try substitution: $T(n)=cn$, RHS is $2cn+\sqrt{n}$, it seems that LHS might be too small. Next reasonable guess is $n^2$ and you happily observe that $n^2$ satisfies $f(n)=4f(n/2)$. 
  \par Possibility 2: try dropping the $\sqrt{n}$ term, then observe that 
  \[f(n)=4f(n/2)=\dots=4^{\log{n}}f(1) = f(1)n^2.\]
  \par In both cases it is then easy to show that $T(n) \ge cn^2$ and hence $T(n) \in \Omega(n^2)$. 
\end{frame}
\begin{frame}{Q5}
  \[T(n)=4T(n/2)+\sqrt{n}\]
  Step 2: prove bounds (in which case, the hard case is the upper bound). 
  \par Most reasonable to try substituting $T(n)=An^2+B\sqrt{n}$. Now you can reverse engineer to make induction work:
  $T(1)=A+B$, 
  \begin{align*}
    T(n) &= 4T(n/2)+\sqrt{n} \\ 
    &\le 4\left(An^2/4+B\sqrt{n/2}\right)+\sqrt{n} \\
    &=An^2+(4B/\sqrt{2}+1)\sqrt{n} \\
    &\le An^2+B\sqrt{n}
  \end{align*}
  where we want the last inequality to hold. We can solve for $B=-\sqrt{2}/(4-\sqrt{2})$ and take A accordingly. Hence $T(n) \in O(n^2)$ and we are done. 
\end{frame}
\begin{frame}{Q5}
  Question: why is it that lower bound is $cn^2$ and upper bound is $An^2+B\sqrt{n}$ for \textbf{negative} $B$?
  \par The `paradox' resolves when you realise that for $f(n)=4f(n/2)$ such that $c := f(1)=T(1)$ we have
  \[f(n) = cn^2\] 
  and
  \[T(n) = \left(c+\frac{\sqrt{2}}{4-\sqrt{2}}\right)n^2-\frac{\sqrt{2}}{4-\sqrt{2}}\sqrt{n}\]
  (basically same induction but we actually have equalities). 
\end{frame}
\begin{frame}{Q6}
  Alternative method: 
  \[T(k, n) = 2T(k/2, n)+\Theta(nk)\]
  \[\frac{T(k, n)}{k} = \frac{T(k/2, n)}{k/2}+\frac{\Theta(nk)}{k}\]
  by telescoping or pushing $\log{k}$ times we get
  \[\frac{T(k, n)}{k} = T(1, n)+\frac{\log{k}\Theta(nk)}{k}. \]
  Same answer, $T(k, n) \in \Theta(nk\log{k})$.
  \[\mathbb{1}\]
\end{frame}

\end{document}

\iffalse
\begin{frame}
  %Together with the facts in Q2, we get
  \begin{theorem}
    For $[f], [g] \in C/{\sim}$, we define $[f] \le [g]$ iff $f(n) \in O(g(n))$ iff $g(n) \in \Omega(f(n))$. Then $\le$ is a well-defined partial order on $C/{\sim}$. %If $f(n) \in o(g(n))$ i.e. $g(n) \in \omega(f(n))$, then $[f]<[g]$ (meaning, $[f] \le [g]$ but $[f] \ne [g]$). 
  \end{theorem}
  \begin{proof}
    Well-definedness follows from lecture result and transitivity. Antisymmetry (if $[f]\le[g], [g]\le[f]$ then $[f]=[g]$) follows from lecture result. The rest follow from Q2.%The only nontrivial part left is that if $f(n) \in o(g(n))$ and $f(n) \in \Theta(g(n))$, then there is some $c>0$ such that 
    %\[cg(n) \le f(n) < cg(n)\]
    %for sufficiently large $n$, contradiction. 
  \end{proof}
  \begin{theorem}
    For $[f], [g] \in C/{\sim}$, we define $[f] <_o [g]$ iff $f(n) \in o(g(n))$ iff $g(n) \in \omega(f(n))$. Then $<_o$ is a well-defined strict partial order on $C/{\sim}$. 
  \end{theorem}
  \begin{proof}
    Well-definedness 
  \end{proof}
\end{frame}
\begin{frame}
  \begin{remark}
    \alert{They are not total orders on $C/{\sim}$}. 
  \end{remark}
  \begin{remark}
    \alert{It is not true that $[f]<[g] \implies f(n) \in o(g(n))$ (and similarly for $\omega$)
  \end{remark}
  Exercise: give counterexamples for the two claims above. (Hint: consider $g$ that `oscillates')
  \par To reiterate about the second remark: you can prove $[f]\ne [g]$ by proving $f(n) \in o(g(n))$, which is just taking limits (as you will see in Q1; similar for $\omega$). But you cannot show $o$ (or $\omega$) by using $O$ (or $\Omega$) and $\ne$, you will have to use limit or definition. 
\end{frame}
\fi